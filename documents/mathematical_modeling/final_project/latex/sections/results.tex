\documentclass[../main.tex]{subfiles}
\providecommand{\main}{..}
%\biblio
%\addbibresource{\main/main.bib}

\begin{document}

There are 1046 donors in any of these files since 2002 that have died, and
of these dead donors 673 gave posthumous donations. This is a large
percentage of donors, so developing good relationships with elderly donors
is a good idea. For our analysis, we removed posthumous donors because the
amount donated was adjusted by year based on date of death; however, we
still have these people's identification numbers, so we could perform a
separate analysis on these people, if desired.

Below is a table providing information on each of the final five clusters,
which include people who have been contacted at least four times, are not
posthumous donators, and donate less than an average of 48000 1982 dollars
per year (more than 98000 current dollars).

\begin{center}
    \begin{tabular}{||c c c c c c||} 
        \hline
        Measure & Cluster1 & Cluster2 & Cluster3 & Cluster4 & Cluster5 \\ [0.5ex]
        \hline\hline
        number & 68 & 12 & 43 & 230 & 1778 \\
        \hline
        adj amt & 5877 & 40689 & 16912 & 1421 & 145 \\ 
        dev amt & 1909 & 4885 & 4389 & 656 &  157 \\
        \hline
        ave freq & 3.36 & 1.43 & 2.52 & 3.41 & 1.56 \\
        dev freq & 5.71 & 0.85 & 2.94 & 4.92 & 2.51 \\
        \hline
        ave email & 0.257 & 0.235 & 0.235 & 0.238 & 0.356 \\
        dev email & 0.207 & 0.245 & 0.176 & 0.186 & 0.250 \\
        \hline
        ave mail & 0.178 & 0.094 & 0.158 & 0.241 & 0.244 \\
        dev mail & 0.151 & 0.110 & 0.172 & 0.170 & 0.188 \\
        \hline
        ave phone & 0.266 & 0.344 & 0.299 & 0.261 & 0.220 \\
        dev phone & 0.154 & 0.200 & 0.182 & 0.173 & 0.193 \\
        \hline
        ave meeting & 0.300 & 0.328 & 0.308 & 0.260 & 0.181 \\
        dev meeting & 0.147 & 0.198 & 0.159 & 0.180 & 0.177 \\
        \hline
        act freq & 1.79 & 1.99 & 1.99 & 1.38 & 0.918 \\
        dev act & 0.89 & 1.15 & 1.22 & 0.89 & 0.734 \\
        \hline
        male percent & 0.69 & 0.75 & 0.56 & 0.60 & 0.55 \\
        female percent & 0.31 & 0.25 & 0.44 & 0.40 & 0.45 \\
        \hline
        deceased percent & 0.103 & 0.333 & 0.163 & 0.065 & 0.014 \\
        \hline
        ave age & 72.9 & 79.0 & 74.6 & 68.3 & 57.1 \\
        dev age & 12.3 & 11.60 & 11.6 & 13.6 & 15.3 \\
        \hline
        alum percent & 85.3 & 58.3 & 51.2 & 72.2 & 74.9 \\
        former parent & 19.1 & 33.3 & 34.5 & 36.5 & 25.8 \\
        \hline
        no solicit & 4.4 & 0.0 & 4.7 & 1.3 & 1.6 \\
        \hline

    \end{tabular}
\end{center}
In the above table, adj amt is the average adjusted yearly amount donated,
ave freq is the average yearly number of donations, the next four averages
are average percentages of contact, act freq is the average number of times
per year contacted, and all statistics starting with dev are standard
deviations. There are many trends that one could spend a lot of time
looking at.

One thing to notice is the particularly large number of females in cluster 3
even though the average age is around 68, during a time when there were
proportionally less women in college. It is likely that there are a lot of
married couples in this group or people in a high-paying gender neutral
field, and this can be easily checked using the ids
of married people that were emailed to us. There is also a large number of
women in cluster four, and both cluster three and cluster four have a large
percentage of former parents, so it is likely that there are a large number
of married couples in these groups, even ones with children. One idea is to
try to get Hendrix students and alumni dating by promoting dating culture on
campus, hosting events for single alumni, or other methods.

Based on the similar ages in groups one through four, and the fact that
people in cluster four are contacted the least and donate the least, it
might be a good idea to contact people in this cluster a little more. They
receive phone calls less frequently than the first three clusters, and more
mail. I think I good plan here would be to tone the mail down slightly and
call them more often. Even better would be to look at them individually and
try to figure out a way to coax them into coming to a meeting or two. 
\end{document}
