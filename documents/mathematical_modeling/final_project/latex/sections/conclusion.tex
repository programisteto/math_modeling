\documentclass[../main.txt]{subfiles}
\providecommand{\main}{..}
%\biblio
%\addbibresource{\main/main.bib}

\begin{document}

This short exercise in data mining turned out to be mostly unfruitful;
however, much of the ground work for future clustering has been done here.
I have processed the data and stored it in connected hash tables and wrote a function you can
pass keys, outliers, and a maximum number of clusters, and it will perform
the elbow method for you using whatever you are clustering over. I have also
made a pretty easy process for extracting the biographical information of
people clustered, but I could work on making it more automated. These tools
make it easy to cluster over different subsets of the data, perhaps to look
for kinds of alumni, parents, etc. I have stored the ids of different
subgroups in lists (parents, former parents, alumni, staff, etc.), and I
could simply pass these into the function that I wrote. Perhaps more
interesting relationships can be found by pulling out all of the
biographical information for clusters and performing an analysis somehow. It
would be interesting to cluster people based on their donation history and
then see if there are any interesting trends in the biographical information
about these cluster. If
the Office of Advancement is interested lists of ids for anyone in these
clusters, posthumous donors, or other categories, just shoot me an email.

\end{document}
