\documentclass[../main.tex]{subfiles}
\providecommand{\main}{..}
%\biblio
%\addbibresource{\main/main.bib}

\begin{document}

The project undertaken for this class was provided by the Office of
Advancement at Hendrix College. The Office of Advancement has been
collecting data on potential donors and event attendees for a while, and the
Office of Advancement gave us this data, hoping that we would be able to
find some useful relationships. The objective of this data mining excursion
was to tell the Office of Advancement how they can alter how they contact
people in order to improve donation and event outcomes. This is important
because contacting people in such a way that they donate more or attend more
events implies that you are contacting these people in a way that they
prefer, in a way that makes them want to be more affiliated with or donate
more resources to the college. Additionally, improving event turnouts is a
good indicator of the overall image of Hendrix. More people at events
means more people want to be there, and it also means that there are more 
possibilities for the college to connect with people on a personal basis,
which could improve future donation and event outcomes.

Clearly, our objective is a noble one, but how to go about accomplishing it is somewhat
tricky. Our idea for accomplishing this consisted of two things. The first
one was just looking around at the data a bit to see if we could find
anything interesting. The second idea, and main focus of analysis in the
paper, is that there are different groups of people who prefer to be
contacted in different ways, and if we can try to put these people into
different groups, we might be able to obtain some insights about how actions
should be altered for each of the groups. To do this, we spent a decent
amount of time looking into clustering algorithms.

\end{document}
