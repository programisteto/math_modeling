\documentclass[../main.txt]{subfiles}

\begin{document}

Talk about K-means clustering algorithm. Used Jupyter Notebook with the
Python 3.0 kernel and the scipy library for clustering. Since the algorithm
is initialized in this way, blah blah blah. It has been proven that
algorithms initialized in this way, blah blah blah, then the algorithm is
less prone to being affected by outliers, meaning the final results of who
is in what cluster is less likely to be affected by outliers in the
clustering data.

Talk about Elbow method for testing. Choosing the correct number of
clusters, $k$, for the clustering algorithm is many times ambigious,
but there are methods for estimating the number of clusters CITE. When it
comes to choosing the right $k$ for the algorithm is a balance between
compression and accuracy. The main method is the Elbow method, which is blah
blah blah.


\biblo
\end{document}
